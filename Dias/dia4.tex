%
% dia4.tex
% 
% Copyright 2014 Rony J. Letona <rony@zronyj.com>
% 
% This program is free software; you can redistribute it and/or modify
% it under the terms of the GNU General Public License as published by
% the Free Software Foundation; either version 2 of the License, or
% (at your option) any later version.
% 
% This program is distributed in the hope that it will be useful,
% but WITHOUT ANY WARRANTY; without even the implied warranty of
% MERCHANTABILITY or FITNESS FOR A PARTICULAR PURPOSE.  See the
% GNU General Public License for more details.
% 
% You should have received a copy of the GNU General Public License
% along with this program; if not, write to the Free Software
% Foundation, Inc., 51 Franklin Street, Fifth Floor, Boston,
% MA 02110-1301, USA.
%

\documentclass[10pt,letterpaper]{article}
\usepackage[latin1]{inputenc}
\usepackage[spanish]{babel}
\usepackage{graphicx}
\usepackage{amsmath}
\usepackage{amsfonts}
\usepackage{amssymb}
\usepackage{color}
\usepackage{float}
\usepackage[left=2cm,right=2cm,top=2cm,bottom=2cm]{geometry}
\author{Rony J. Letona}
\title{Taller de Qu\'imica Computacional Aplicada: D\'ia 4}
\definecolor{light-gray}{gray}{0.90}

\newcommand{\tab}[1]{\hspace{.2\textwidth}\rlap{#1}}

\newcommand{\inlinecode}[1]{
\colorbox{light-gray}{\texttt{#1}}
}

\newsavebox{\selvestebox}
\newenvironment{Code}
{
\begin{lrbox}{\selvestebox}%
\begin{minipage}{\dimexpr\columnwidth-2\fboxsep\relax}
\fontfamily{\ttdefault}\selectfont
}
{\end{minipage}\end{lrbox}%
\begin{center}
\colorbox{light-gray}{\usebox{\selvestebox}}
\end{center}
}

\newcommand{\Picture}[1]
{
	\begin{figure}[H]
	\begin{flushleft}
	\includegraphics[width=\columnwidth]{#1}
	\end{flushleft}
	\end{figure}
}

\begin{document}
\maketitle

\section{Ejercicios con \LaTeX\ }
Los reportes son una de las cosas con las que m\'as nos topamos tanto en un ambiente acad\'emico, como en uno laboral. Si no estamos tratando de explicar por qu\'e no nos sali\'o el producto en la s\'intesis, estamos buscando c\'omo ligar la evidencia de un an\'alisis y que eso nos lleve a una explicaci\'on. Sin embargo, siempre debemos documentar cada paso. Sin embargo, hay una cosa que suele quitarnos m\'as de nuestro preciado tiempo: el formato. Cuando comenz\'abamos a hacer reportes en los primeros a\~nos fuimos ingenuos y los hicimos todos desde cero. Adem\'as, muchas veces nos los pidieron a hechos a mano. Luego ya no fue as\'i, pero cada instructor o catedr\'atico nos los ped\'ia diferentes. Al final de cuentas siempre result\'abamos haciendo una plantilla para ese laboratorio y la segu\'iamos usando todo el semestre. Si bien nos iba, la plantilla era buena y agregarle cosas no nos arruinaba cosas como la numeraci\'on de las p\'aginas o la colocaci\'on de las im\'agenes, pero si no, nos iba a tomar unos 20 minutos arreglar el problema. Adem\'as de esto, ni nuestros proyectos, ni nuestros reportes se vieron profesionales. Acept\'emoslo, nunca se hab\'ian visto como un art\'iculo cient\'ifico, como un peque\~no manual o como un libro. ``Eso es trabajo de una editorial.''--pensar\'ian algunos. Pero hoy vamos a ver que no, y que hacer trabajos con un acabado de editorial no es nada complicado.\\

Hace ya varios a\~nos se intent\'o crear un programa que facilitara la impresi\'on de documentos en cualquier tipo de impresora; desde impresoras de matriz hasta impresoras laser. Este programa se hizo llamar \TeX\ y su desarrollo llev\'o a que sea utilizado por editoriales grandes hasta nuestros d\'ias. El ``dialecto'' m\'as utilizado de \TeX\ hoy d\'ia es llamado \LaTeX\ . Este result\'o siendo el lenguaje de programaci\'on adoptado por la comunidad cient\'ifica en muchas partes por sus ventajas para crear plantillas, escribir ecuaciones, numerarlas, insertar bibliograf\'ias, autonumerar p\'aginas, etc. Se dice que es un lenguaje de programaci\'on como HTML, pues se basa en peque\~nas palabras indicando las diferentes partes del documento. Y la idea final es obtener un documento que no solo tenga un contenido ideal, como solemos inclu\'irselos a nuestros proyectos, sino que se vea excepcional sin tener que invertir mucho tiempo en darle formato. Pero para qu\'e explicar lo bien que se puede ver un documento de estos? Es un poco redundante. El documento que estamos viendo fue hecho en \LaTeX\ , al igual que el de ayer, anteayer y el de los dem\'as d\'ias. Pero la idea no es que lo sepamos identificar en un documento, sino que lo sepamos usar. Comencemos, pues, a crear nuestros documentos de una forma diferente!

\subsection{Comenzando con un Proyecto}
Para trabajar con \LaTeX\ suele no hacer falta nada m\'as que un editor de texto y saber c\'omo crear un archivo \emph{PDF} a partir del archivo \TeX . Sin embargo, siendo \LaTeX\ tan extenso, vamos a utilizar un editor en espec\'ifico. Con este podremos ahorrarnos muchos problemas a la hora de crear nuestros documentos. Como primer paso, vamos a abrir \textbf{\TeX Maker} y all\'i vamos crear un documento nuevo. Este nuevo archivo en blanco lo vamos a guardar como \emph{proyecto\_01.tex}. Ahora vamos a iniciar con nuestro proyecto.\\

Vamos a iniciar el documento escribiendo:
\begin{Code}
\begin{verbatim}
\documentclass[12pt,letterpaper]{article}

\begin{document}

\end{document}
\end{verbatim}
\end{Code}

Aqu\'i iniciamos especificando el tama\~no de letra que se utilizar\'a en todo el documento, y el tama\~no de p\'agina con el que se trabajar\'a. Finalmente, en la primera l\'inea vemos que se especifica que vamos a crear un art\'iculo. Hay otras opciones como \inlinecode{letter}, \inlinecode{report} y \inlinecode{book}. La diferencia entre ellos son los comandos y la forma de aplicar el formato (el libro utiliza cap\'itulos, la carta solicita encabezados, fechas y firma, el reporte es como un informe final de proyecto, etc). Luego especificamos en d\'onde deseamos tener el contenido del documento: entre el inicio y el final del documento. All\'i dentro vamos a poner el contenido del archivo \emph{lorem\_ipsu.txt} (c\'opialo y p\'egalo). Luego, en la barra superior de \textbf{\TeX Maker} vas a hallar dos flechas azul claro. A la par de la primera debe de decir PDFLaTeX y a la par de la segunda debe decir View PDF. La idea all\'i es tener configurada la parte que nos permitir\'a transformar nuestro archivo \TeX\ en un PDF y poder visualizarlo despu\'es.

\Picture{img/tex_01.png}

Cuando esas dos cosas ya est\'en como en la im\'agen, podemos presionar la primera flecha azul. En la barra de abajo deber\'ia de aparecer algo que nos diga que el proceso ha terminado exitosamente. Ahora presionaremos la segunda flecha azul y el resultado nos ser\'a revelado: un documento escrito con un estilo bastante limpio y hasta con p\'aginas numeradas. Genial! Ahora ... hay ciertas cosas que no se ven muy bien. En primer lugar, vamos a agregar espacios entre p\'arrafos. Para ello, agregaremos dos diagonales inversas \inlinecode{\textbackslash \textbackslash } al terminar de cada p\'arrafo: despu\'es del punto y volvemos a presionar ambas flechas.\\

Ahora ya separamos los p\'arrafos y nuestro documento ya se ve mejor. Antes de proceder a la siguiente secci\'on repasemos en lo que hicimos. Inicializamos un documento especificando un tama\~no de letra y de p\'agina. Luego colocamos algo de texto dentro de \'el y finalmente separamos los p\'arrafos con una diagonal doble inversa. No es tan complicado, aunque s\'i requiere un poco de tiempo. Pero lo m\'as importante: no se ve como nosotros queremos que se vea. Para ello, vamos a continuar.

\subsubsection{Geometr\'ia}

Hasta ahora, si pensamos en c\'omo se ve nuestro documento, vamos a notar que los m\'argenes son muy grandes. Generalmente esto es algo que solo cambiamos cuando deseamos que la cantidad de p\'aginas en nuestros proyectos sea mayor o menor. En este caso, sin embargo, estamos desperdiciando demasiado espacio en los m\'argenes. C\'omo corregimos eso? La forma de cambiar m\'argenes en \LaTeX\ es m\'as sencilla de lo que uno se imagina, y suele hacerse siempre al inicio de los documentos que vayamos a producir. En nuestro caso, dejaremos m\'argenes de $2cm$ mediante el uso del paquete \inlinecode{geometry}. El resultado ser\'a algo as\'i:

\begin{Code}
\begin{verbatim}
\documentclass[12pt,letterpaper]{article}
\usepackage[left=2cm,right=2cm,top=2cm,bottom=2cm]{geometry}

\begin{document}

Lorem ipsum dolor ...

\end{document}
\end{verbatim}
\end{Code}

Al volver a correr esto con las dos flechas azules, nos damos cuenta de que nuestro resultado es mucho mejor. Ya se nota que tiene un toque de documento. Y la mejor parte es que \LaTeX\ usa sangr\'ias en cada p\'arrafo y los deja justificados a\'un utilizando guiones! Se ve bastante bien.

\subsubsection{Idiomas}

Ahora agregaremos un par de paquetes m\'as que nos permitir\'an escribir lo que nosotros querramos dentro de nuestro documento. Y es que \LaTeX\ fue dise\~nado originalmente para el idioma ingl\'es. Un ejemplo de eso es que \textbf{las tildes} y \textbf{la \~n} no las podemos escribir f\'acilmente. Pero m\'as que eso, los guiones para dividir una palabra son importantes. Para ello, vamos a agregar otro paquete en el documento que llevamos.

\begin{Code}
\begin{verbatim}
\documentclass[12pt,letterpaper]{article}
\usepackage[left=2cm,right=2cm,top=2cm,bottom=2cm]{geometry}
\usepackage[spanish]{babel}

\begin{document}

Lorem ipsum dolor ...

\end{document}
\end{verbatim}
\end{Code}

Con esto ya muchas cosas se nos permiten en espa\~nol. M\'as adelante vamos a ver qu\'e impacto tiene realmente eso. Ahora, vamos a ver dos alternativas para lograr escribir las tildes y la \~n. La forma \emph{complicada} de escribirlas en \LaTeX\ es as\'i:

\begin{Code}
\begin{verbatim}
\'a
\'e
\'i
\'o
\'u
\~n
\end{verbatim}
\end{Code}

Ahora, a pesar de que parece bastante tedioso escribir de esa manera, es la \'unica infalible. Aunque el documento est\'e configurado para ingl\'es, esto garantiza que podamos tener tildes y \~ns. La otra forma de hacerlo sin complicarnos tanto la vida es incluyendo otro paquete en nuestro documento: un conjunto de caracteres universal.

\begin{Code}
\begin{verbatim}
\documentclass[12pt,letterpaper]{article}
\usepackage[left=2cm,right=2cm,top=2cm,bottom=2cm]{geometry}
\usepackage[utf8]{inputenc}
\usepackage[spanish]{babel}

\begin{document}

Lorem ipsum dolor ...

\end{document}
\end{verbatim}
\end{Code}

Con este paquete ya no necesitamos estar usando el gui\'on y las ap\'ostrofes, sino que podemos escribir libremente y no habr\'an problemas. En este caso en particular, estos documentos est\'an escritos de la primera manera para garantizar que quien no posea el paquete en su ordenador, pueda de todas formas armar estos documentos. Sin embargo, la brecha de idiomas ya la cubrimos. Ahora solo vamos a profundizar en una \'ultima cosa: paquetes.\\

\subsubsection{Paquetes}

Los paquetes son formas de extender las funciones normales de \LaTeX . Como ya hemos visto, usamos paquetes para los m\'argenes, para el conjunto de caracteres permitidos y para idioma. Existen paquetes para muchas otras cosas m\'as. La sintaxis para incluirlos, sin embargo, es siempre la misma: \inlinecode{\textbackslash usepackage\{\emph{paquete}\}}\\

Todav\'ia vamos a ver aqu\'i un paquete m\'as: encabezados y pies de p\'agina. Este nos dar\'a una idea de qu\'e hacer y c\'omo darle una forma m\'as elegante a nuestros documentos.\\

\subsubsection{Encabezado y Pie}

El encabezado y pie de p\'agina suele ser todo un rollo en Microsoft Word. Estos suelen necesitar mucha configuraci\'on y bastante paciencia. En \LaTeX\ el asunto no es tan diferente, pero veremos que existen algunas diferencias que nos facilitar\'an el proceso. Comencemos importando el paquete que nos permitir\'a hacer el encabezado y el pie.\\

\begin{Code}
\begin{verbatim}
\documentclass[12pt,letterpaper]{article}
\usepackage[left=2cm,right=2cm,top=2cm,bottom=2cm]{geometry}
\usepackage[utf8]{inputenc}
\usepackage[spanish]{babel}
\usepackage{fancyhdr}
\pagestyle{fancy}

\begin{document}

Lorem ipsum dolor ...

\end{document}
\end{verbatim}
\end{Code}

Si ponemos atenci\'on, notaremos que no solo incluimos un paquete, sino que le pedimos un estilo particular a la p\'agina. Lo que hace esto es que construye un estilo de encabezado y pies de p\'agina basados en el n\'umero de la p\'agina y en la secci\'on o cap\'itulo que se halle en esa p\'agina. Ahorita aunque re-armemos nuestro PDF no lo notaremos (solo notaremos una delgada l\'inea en el encabezado), pero en la siguiente secci\'on esto se har\'a evidente.

\subsection{Decorando el Documento}
\subsubsection{T\'itulo}
Ahora que ya tenemos un formato general para nuestro documento, lo vamos a comenzar a decorar un poco. Lo primero que vamos a agregarle ser\'a un t\'itulo. Pero para eso necesitamos especificar 2 cosas: el autor y el t\'itulo de nuestro trabajo. En este caso, el autor vamos a ser nosotros mismos y el t\'itulo ser\'a: Lorem Ipsum. El c\'odigo se ver\'a ahora algo as\'i:

\begin{Code}
\begin{verbatim}
\documentclass[12pt,letterpaper]{article}
\usepackage[left=2cm,right=2cm,top=2cm,bottom=2cm]{geometry}
\usepackage[utf8]{inputenc}
\usepackage[spanish]{babel}
\usepackage{fancyhdr}
\pagestyle{fancy}
\author{Mi Nombre}
\title{Lorem Ipsum}

\begin{document}

\maketitle

Lorem ipsum dolor ...

\end{document}
\end{verbatim}
\end{Code}

Notemos que dentro del documento incluimos una peque\~na palabra nueva: \inlinecode{\textbackslash maketitle} Es justo aqu\'i que el t\'itulo es incluido dentro de nuestro documento. Si no lo incluimos, ninguna de las otras 2 entradas tienen sentido. El resultado, al armar nuestro documento en PDF es algo vistoso, pero cumple con lo que dese\'abamos. Adem\'as, incluye la fecha actual en espa\~nol (otra cosa que sale de usar el paquete para espa\~nol). Es importante darnos cuenta de que al incluir el t\'itulo, el encabezado ya no apareci\'o en esa p\'agina! Y hablando de encabezados, vamos a pasar a otra cosa que nos sirve de mucho al elaborar un documento nuevo.

\subsubsection{Secciones y Cap\'itulos}
Dividir un documento en diferentes partes suele hacerlo m\'as f\'acil de leer adem\'as de hacerlo m\'as f\'acil de referenciar. \LaTeX\ suele requerir que nosotros indiquemos d\'onde hay cap\'itulos y/o secciones para poder as\'i armar \'indices o sacar referencias. Antes de comenzar vamos a aclarar algo: Las secciones pueden ser utilizadas en cualquier documento. Los cap\'itulos solo pueden ser usados en reportes o libros.\\

En este caso estamos trabajando con un art\'iculo, as\'i que vamos a agregarle diferentes secciones a nuestro documento. Pero para que se note la diferencia, vamos a agregar m\'as texto. Vamos a abrir todos los documentos \emph{ipsum} en nuestro directorio y los vamos a agregar con sus respectivos nombres como secciones (recuerda separar los p\'arrafos):

\begin{Code}
\begin{verbatim}
\documentclass[12pt,letterpaper]{article}
\usepackage[left=2cm,right=2cm,top=2cm,bottom=2cm]{geometry}
\usepackage[utf8]{inputenc}
\usepackage[spanish]{babel}
\usepackage{fancyhdr}
\pagestyle{fancy}
\author{Mi Nombre}
\title{Lorem Ipsum}

\begin{document}

\maketitle

\section{Lorem Ipsum}
Lorem ipsum dolor sit amet ...

\section{Bacon Ipsum}
Bacon ipsum dolor sit amet corned beef ...

\section{Cupcacke Ipsum}
Cupcake ipsum dolor sit amet candy canes ...

\section{Hipster Ipsum}
Before they sold out swag Pitchfork roof ...

\section{Tuna Ipsum}
Squaretail; ghost flathead hatchetfish ...

\section{Veggie Ipsum}
Veggies es bonus vobis, proinde vos postulo ...

\end{document}
\end{verbatim}
\end{Code}

Ahora s\'i que se ha vuelto grande nuestro documento! Pero m\'as que eso, notemos algo importante: las \emph{secciones} se colocan como t\'itulos exactamente donde las pusimos. Adem\'as de esto, el encabezado toma ahora el nombre de la \textbf{\'ultima} secci\'on que comienza en esa p\'agina en particular. Es recomendable bajarle el tama\~no de letra a 11 al documento para visualizar mejor estos cambios.\\

Un detalle que vale la pena mencionar es que las secciones no necesariamente tienen que tener un n\'umero. Si creamos una secci\'on mediante \inlinecode{\textbackslash section\{\emph{Nombre de la Secci\'on}\}} esta va a aparecer numerada. Si en vez de eso la creamos mediante \inlinecode{\textbackslash section*\{\emph{Nombre de la Secci\'on}\}} esta no aparecer\'a numerada, sino que solo aparecer\'a el t\'itulo de la misma.\\

En un documento de regular tama\~no, tener secciones a veces no es suficiente. Necesitamos tener divisiones adentro de las secciones. Para este tipo de casos, \LaTeX\ ofrece las sub-secciones, las sub-sub-secciones y las sub-sub-sub-secciones (estas \'ultimas no siempre van a comportarse como se esperar\'ia). Estas obedecen los mismos principios que las secciones: se autonumeran, se colocan como t\'itulos de un segmento del documento, y si se les agrega el asterisco en su creaci\'on, la numeraci\'on no aparecer\'a. Estas se crean de la siguiente manera:

\begin{Code}
\begin{verbatim}
\section{Nombre de la seccion}
\subsection{Nombre de la sub-seccion}
\subsubsection{Nombre de la sub-sub-seccion}
\subsubsubsection{Nombre de la sub-sub-sub-seccion}
\end{verbatim}
\end{Code}

Usa cada nivel sabiamente. Dividir un documento es bueno, pero dividirlo demasiado puede llevar a problemas de lectura o a que el contenido se vea disminuido. Ahora, ya que hemos visto c\'omo crear, darle forma y escribir en un documento \TeX , vamos a pasar a otro caso un poco diferente. Para esto guarda el trabajo que tengas, ci\'erralo y abre un documento nuevo en \textbf{\TeX Maker}. Ahora vamos a ver lo realmente bonito de \LaTeX .

\subsection{Notaci\'on Cient\'ifica}
Una de las caracter\'isticas tan especiales de \LaTeX\ es que se puede utilizar para ciencia. No solo por su formalidad y transparencia a la hora de crear documentos (no suceden cosas raras porque el programa lo quiso as\'i), sino porque se puede incluir notaci\'on cient\'ifica en ellos. Esta es una de la razones principales por las que se prefiere utilizar este lenguaje para textos cient\'ificos.\\

Para comenzar, vamos a hacer que nuestro nuevo documento sea un art\'iculo de letra tama\~no 10, m\'argenes sin alterar, en espa\~nol, con un grupo de caracteres que acepte tildes y \~n, sin encabezados bonitos, pero con un t\'itulo: \emph{Notaci\'on Cient\'ifica} y escrito nada m\'as que por nosotros. Cuando ya tengamos esto listo, podemos proceder a seguir leyendo.\\

Bueno, una vez ya tengamos listo un documeto as\'i, vamos a agregarle unos cuantos paquetes extra a nuestro documento. Estos paquetes \textbf{NO} son indispensables, pero ayudan mucho cuando se trata de notaci\'on cient\'ifica. Estos son:

\begin{Code}
\begin{verbatim}
\usepackage{amsmath}
\usepackage{amsfonts}
\usepackage{amssymb}
\end{verbatim}
\end{Code}

\noindent Al agregar esto, nuestro c\'odigo ha de verse algo as\'i:

\begin{Code}
\begin{verbatim}
\documentclass[10pt,letterpaper]{article}
\usepackage[utf8]{inputenc}
\usepackage[spanish]{babel}
\usepackage{amsmath}
\usepackage{amsfonts}
\usepackage{amssymb}
\author{Mi Nombre}
\title{Notaci\'on Cient\'ifica}

\begin{document}

\maketitle

\end{document}
\end{verbatim}
\end{Code}

Ahora podemos comenzar a utilizar la notaci\'on cient\'ifica en nuestro documento. Como esta puede comprender muchos \'areas, vamos a ir paso a paso como cuando se comienza a aprender matem\'atica. Lo primero que vamos a aprender es a preparar un lugar en nuestro documento d\'onde colocar esta notaci\'on. Si deseamos que los s\'imbolos o notaci\'on aparezcan dentro del texto, como esto: $H_2 CO_3 \longrightarrow^{\hspace{-0.4cm} \Delta}\hspace{0.2cm} H_2 O + CO_2$, entonces utilizamos los signos de dolar al iniciar y finalizar una expresi\'on: \inlinecode{\$ \emph{expresi\'on} \$}. A esta forma de notaci\'on se le llama \emph{inline}. La otra forma de incluir notaci\'on cient\'ifica es declarando un segmento como de ecuaci\'on de esta manera:

\begin{Code}
\begin{verbatim}
\begin{equation}

expresion

\end{equation}
\end{verbatim}
\end{Code}

\noindent A un segmento as\'i en \LaTeX\ (declarado con un principio y un final) se le llama \emph{ambiente}. El resultado se ver\'a as\'i:

\begin{equation}
H_2 CO_3 \longrightarrow^{\hspace{-0.4cm} \Delta}\hspace{0.2cm} H_2 O + CO_2
\end{equation}

Algo que podemos notar de inmediato es que la forma en que se muestra la reacci\'on es ahora m\'as limpia, centrada y (muy importante) numerada. S\'i, \LaTeX\ autonumera nuestras ecuaciones, reacciones o f\'ormulas por defecto. Esto nos permite referirnos a ellas dentro del documento posteriormente. Vamos a crear una nueva secci\'on dentro de nuestro documento de notaci\'on cient\'ifica a la que llamaremos \emph{Operaciones B\'asicas}. En ella vamos a explicarle a alguien m\'as c\'omo se suma, resta, multiplica y divide en \LaTeX . La descripci\'on es cosa nuestra; nos toca ponernos creativos. Lo que s\'i vamos a ver es c\'omo se hacen estas operaciones. Probemos irlas incluyendo dentro de nuestra explicaci\'on de manera \emph{inline} para mantener la continuidad del texto.

\begin{Code}
\begin{verbatim}
Sumar:       $ A + B = C $
Restar:      $ D - E = F $
Multiplicar: $ G \cdot H = I $
Dividir:     $ \frac{J}{K} = L $
\end{verbatim}
\end{Code}

Al final generemos nuestro PDF para ver c\'omo se ve el resultado. Ahora mostr\'emoselo a nuestro compa\~nero de al lado y discutamos sobre lo que ambos creen que hace cada parte de lo que hemos visto. Una vez terminado esto, vamos a pasar a otra parte m\'as intensa. Vamos a crear otra nueva secci\'on de nuestro documento llamada \emph{\'Algebra}. Aqu\'i vamos a explicar que el uso de notaci\'on cient\'ifica dentro de un texto lo puede hacer m\'as comprensible al lector. Como ejemplo vamos a proponer la f\'ormula de Vieta, la cual nos sirve para calcular las dos ra\'ices de un polinomio de grado dos.

\begin{equation}
x_{1,2} = \frac{-B \pm \sqrt{B^2 - 4 A C}}{2 A}
\end{equation}
 \\

Y que en el caso de que $ B^2 - 4 A C > 0 $ entonces $ x_{1,2} \in \mathbb{R} $, o que en el caso en que $ B^2 - 4 A C = 0 $ entonces $ x \in \mathbb{R} $. Pero si $ B^2 - 4 A C < 0 $ entonces $ x_{1,2} \in \mathbb{C} $. Esta explicaci\'on quiz\'a nos tome un rato, pero para ayudarnos se nos dej\'o aqu\'i la manera en que se representan cada una de estas operaciones y relaciones.

\begin{Code}
\begin{verbatim}
Potencias:     $ A^{3} = A \cdot A \cdot A $
Raiz cuadrada: $ \sqrt{B} = B^{1/2} $
Raiz radical:  $ \sqrt[n]{C} = C^{1/n} $
Subindices:    $ D_{2} = $ deuterio molecular
Mayor igual:   $ E \geq F $
Menor igual:   $ G \leq H $
Mayor y menor: $ I > J < K | I = 1, K = 2, J = -5 $
Mas-menos:     $ \sqrt{4} = \pm 2 $
Conjuntos:     $ \mathbb{N}, \mathbb{Z}, \mathbb{Q}, \mathbb{I}, \mathbb{R}, \mathbb{C} $
Pertenece a:   $ 1, 2, 3, \ldots \in \mathbb{N} $
Para todo:     $ \forall n \in \mathbb{N} | n^{2} \geq n $
No existe:     $ \frac{x}{0} = \nexists $
\end{verbatim}
\end{Code}

Se nos recomienda probar cada una de las expresiones que se nos han dado antes de utilizarla dentro de nuestra explicaci\'on; esto nos dar\'a una mejor idea de c\'omo se ver\'a el resultado o qu\'e significa cada secuencia, palabra y signo. Como antes, al terminar nuestra explicaci\'on, vamos a mostr\'arsela a nuestro compa\~nero de a la par y vamos a corroborar que nuestras explicaciones sean correctas. Por cierto, la secuencia \inlinecode{\textbackslash ldots} es la que nos da tres puntos suspensivos.\\

Antes de pasar a la siguiente secci\'on hay un peque\~no truco que utilizamos en \'algebra que no hemos visto todav\'ia: par\'entesis. Los par\'entesis no son algo complicado en \LaTeX , sin embargo, s\'i se debe saber c\'omo llamarlos, pues de hacerlo mal estos no se comportar\'an como nosotros deseamos. Lo \'unico importante que debemos tomar en cuenta a la hora de usar par\'entesis es lo siguiente:\\
\noindent\inlinecode{\textbackslash left\#} declara un par\'entesis que ir\'a a la izquierda\\
\noindent\inlinecode{\textbackslash right\#} declara un par\'entesis que ir\'a a la derecha\\

Claro que en vez del numeral \inlinecode{\#} debe ir el tipo de par\'entesis que deseamos. Los ejemplos claros est\'an a continuaci\'on.

\begin{Code}
\begin{verbatim}
(a + b):        $ \left( a + b \right) $
[x, y]:         $ \left[ x , y \right] $
<i, j>:         $ \left\langle i , j \right\rangle $
{1, 2, 3, ...}: $ \left\lbrace 1 , 2 , 3 , \ldots \right\rbrace $
\end{verbatim}
\end{Code}

Debemos de tener especial atenci\'on con las llaves y los par\'entesis angulares, porque estos no utilizan un signo en vez del numeral, sino otra secuencia. Fuera de eso, la idea en todos los casos es igual. Agreguemos esto a nuestra explicaci\'on en nuestro documento sobre notaci\'on cient\'ifica y discutamos con otro compa\~nero sobre nuestras conclusiones.\\

A continuaci\'on vamos a crear una secci\'on m\'as en nuestro documento. Esta se llamar\'a \emph{C\'alculo}. En esta secci\'on vamos a profundizar en los \'ultimos toques de matem\'atica. En nuestro documento vamos a explicar dos casos en particular, y luego nos vamos a extender con algunos detalles extra. Lo primero que vamos a explicar es c\'omo escribir la definici\'on de una derivada.

\begin{equation}
f' \left( x \right) = \lim_{h \to \infty} \frac{f \left( x + h \right) - f \left( x \right) }{h} = \dfrac{d}{d x} f \left( x \right)
\end{equation}
 \\

Luego, para complementar un poco m\'as las cosas, pasaremos a explicar c\'omo se escribe una integral desde su definici\'on. Recordemos que $\Delta x = \frac{b - a}{n}$ y que $x_{i}^{*} = a + (\Delta x) \cdot i$. Luego procedemos con la definici\'on:

\begin{equation}
A = F \left( b \right) - F \left( a \right) = \lim_{n \to \infty} \sum_{i = 1}^{n} \left( f\left(x^{*}_{i}\right) \cdot \Delta x \right) = \int_{a}^{b} f \left( x \right) dx
\end{equation}
 \\

Otra vez resultamos con cosas que nos parecen un poco monstruosas, pero no nos dejemos enga\~nar por pas apariencias. La mayor parte de lo que tenemos aqu\'i ya lo sabemos hacer. Lo \'unico nuevo son: los l\'imites, la derivada, la sumatoria y la integral. Esas se construyen de la siguiente manera:

\begin{Code}
\begin{verbatim}
Sumatoria: $ \sum_{ i = 0 }^{ n } i = \frac{ n \left( n + 1 \right) }{ 2 } $
Limite:    $ \lim_{ x \to 0^{+} } \log(x) = - \infty  $
Derivada:  $ \dfrac{ d }{ d x } f = f' $
Integral:  $ \int_{ - \infty }^{ \infty } e^{ -x^{2} } dx = \sqrt{ \pi } $
\end{verbatim}
\end{Code}

A pesar de que se mostraron entre los signos de dolar, es \textbf{muy} recomendable que estos signos no se utilicen como texto \emph{inline}, sino que dentro del ambiente de ecuaciones; all\'i se aprecian mejor. Para cuando hagamos nuestras pruebas, intentemos hacerlo as\'i. Ahora, entre las nuevas secuencias que hemos visto, la de derivada es una que se parece mucho a la de fracciones, pero tenemos que ponerle atenci\'on a esa \emph{d}; tiene su raz\'on de ir all\'i. Por otra parte, qu\'e pasar\'ia si nuestra derivada es parcial? Pues usamos \inlinecode{\textbackslash partial} en vez de \inlinecode{d} Y si deseamos hacer derivadas en varias dimensiones de una vez, podemos utilizar \inlinecode{\textbackslash nabla}. De esta forma ya podemos escribir cosas como:

\begin{equation}
\nabla^2 \psi = \dfrac{\partial^2 \psi}{\partial x^2} + \dfrac{\partial^2 \psi}{\partial y^2} + \dfrac{\partial^2 \psi}{\partial z^2}
\end{equation}

Esto \'ultimo nos lleva a considerar algo: todas las letras griegas existen en \LaTeX . Lo que hay que saber para utilizarlas es su secuencia, que en este caso es el nombre de cada una de ellas. Cuando se desean min\'usculas, la secuencia va toda en min\'usculas. Cuando se desea una may\'uscula, la primera letra de la secuencia debe escribirse en may\'uscula.\\

Finalmente, vamos a crear una secci\'on m\'as en nuestro documento: \emph{Qu\'imica}. Lo primero que hicimos en este segmento vamos a intentar repetirlo aqu\'i: la reacci\'on de descomposici\'on del \'acido carb\'onico. Antes de que nos den el c\'odigo con el que se hace eso, propongamos una forma de hacerlo con lo que ya sabemos.\\

Comencemos con lo sencillo: no hay necesidad especial de caracteres extra\~nos, solo sub\'indices. Otra cosa que es conveniente es que para dejar espacios en blanco, generalmente solo ponemos una diagonal inversa y dejamos un espacio despu\'es de ella: \inlinecode{\textbackslash  } Y lo m\'as complicado ser\'a, por ahora, la flecha que indica reacci\'on \inlinecode{$\longrightarrow$} Hemos de notar que esta flecha es m\'as grande que la utilizada en los l\'imites. Ahora intentemos escribir la reacci\'on anterior, pero especificando en qu\'e estado se encuentra cada sustancia tambi\'en.\\

El c\'odigo para este caso se ver\'a as\'i. Aqu\'i falta un detalle: el signo de calor. Este, aunque suene sencillo, es un poco m\'as dif\'icil de colocar, puesto que implica mover un super\'indice hacia atr\'as y mantener el espacio entre la flecha y el texto que la sucede. Se nos provee ac\'a el c\'odigo para colocar el peque\~no tri\'angulo, que de hecho es la letra griega delta $\Delta$.
\begin{Code}
\begin{verbatim}
H_{2}CO_{3\ (ac)} \longrightarrow H_{2}O_{(l)} + CO_{2\ (g)}
H_{2}CO_{3\ (ac)} \longrightarrow^{\hspace{-0.4cm} \Delta}\hspace{0.2cm} H_{2}O_{(l)} +
CO_{2\ (g)}
\end{verbatim}
\end{Code}

Estudiemos el c\'odigo, comentemos con nuestra pareja y resolvamos las dudas necesarias. Luego, tomemos nota en nuestro documento describiendo los aspectos que consideremos importantes. Vamos a ver ahora otros ejemplos. Uno en particular que nos puede servir para cuando estamos desarrollando una ecuaci\'on es el ambiente \emph{Arreglo de Ecuaciones}. En este, uno puede ir escribiendo los pasos en la resoluci\'on de una ecuaci\'on, o los pasos en una demostraci\'on matem\'atica, por ejemplo. El ambiente comienza y termina de la siguiente forma:

\begin{Code}
\begin{verbatim}
\begin{eqnarray}
primera & entrada\\
segunda & entrada\\
tercera & entrada y final
\end{eqnarray}
\end{verbatim}
\end{Code}

Como podemos darnos cuenta, aqu\'i vamos ingresando diferentes pasos y utilizamos la doble diagonal \inlinecode{\textbackslash \textbackslash} para pasar a la siguiente expresi\'on. El signo \inlinecode{\&} lo utilizamos para separar columnas. Esto puede ser \'util para mantener a todas las ecuaciones alineadas sobre el signo \inlinecode{=}. Intentemos algo de mucha utilidad en el \'area de fisicoqu\'imica: cin\'etica de reacci\'on. No entraremos a detalle en el c\'omo, por qu\'e o qu\'e significa lo que vamos a escribir. Solo es un ejercicio para mostrar c\'omo funciona el ambiente de arreglo de ecuaciones. Comencemos!

\begin{Code}
\begin{verbatim}
\begin{eqnarray}
- \dfrac{d \left[ A  \right]}{d t} & = & \kappa \left[ A  \right]\\
- \frac{d \left[ A  \right]}{\left[ A  \right]} & = & \kappa\ dt\\
\int_{ \left[ A  \right]_{0} }^{ \left[ A  \right] } \frac{d \left[ A  \right]}{\left[ A 
\right]} & = & - \kappa \int_{t_{0}}^{t} dt\\
\ln \left( \left[ A  \right] \right) - \ln \left( \left[ A  \right]_{0} \right) & = &
- \kappa \left( t - t_{0} \right)\\
\ln \left( \frac{\left[ A  \right]}{\left[ A  \right]_{0}} \right) & = & - \kappa\ t\\
\frac{\left[ A  \right]}{\left[ A  \right]_{0}} & = & e^{ - \kappa t }\\
\left[ A  \right] & = & \left[ A  \right]_{0} e^{ - \kappa t }
\end{eqnarray}
\end{verbatim}
\end{Code}

Intentemos insertar esto en nuestro documento y discutir cada paso. Ahora se ve monstruoso, pero una vez lo veamos ya procesado en \LaTeX , creo que le perderemos un poco el miedo. Siempre recordemos que cada paso debe terminar con la doble diagonal inversa. En el c\'odigo que aqu\'i se nos da eso no es tan evidente, pues para que quepa el texto, este se ha dividido en m\'as l\'ineas, pero estas deben de ser una sola al estar en el documento.\\

Ahora que ya sabemos bastante sobre c\'omo escribir en notaci\'on cient\'ifica, vamos a intentar un par de ejercicios antes de pasar a la siguiente secci\'on. Intentemos escribir el c\'odigo para las siguientes tres ecuaciones:

\begin{equation}
\Delta G^{o} = \Delta H^{o} - T\ \Delta S^{o} = - RT \ln \left( K_{eq} \right)
\end{equation}

\begin{equation}
CH_{3}(CO)CH_{3} + CH_{3}CH_{2}O^{-} \longrightarrow^{\hspace{-0.4cm}\Delta}\hspace{0.2cm} CH_{3}CO^{-}CH_{2} + CH_{3}CH_{2}OH
\end{equation}

\begin{equation}
\left[ -\frac{\hbar^{2}}{2m} \nabla^{2} + V \left( \textbf{r}, t \right) \right] \Psi \left( \textbf{r}, t \right) = i \hbar \dfrac{\partial}{\partial t} \Psi \left( \textbf{r}, t \right)
\end{equation}
 \\

Consejo que se nos da para esto: No le tengas miedo a una ecuaci\'on porque sus signos son extra\~nos o porque su tama\~no es grande. Generalmente solo se trata de letras que alguien us\'o como nombre para alguna cantidad, y signos que ayudan a no escribir tanto. Lo importante de ellas es saber qu\'e significan y que, como toda ecuaci\'on, fue hecha para manipularla. Tom\'emonos el tiempo para hacer cada cosa, no nos urge un resultado inmediato. Luego pasamos a la siguiente secci\'on.

\subsection{Im\'agenes y Tablas}


\subsection{Art\'iculo Cient\'ifico}
\subsection{Autoensablados: \'Indice y Bibliograf\'ia}
\subsection{Tesis}

\subsection{Comentarios Finales}

\section{Glosario de comandos sencillos}

\begin{small}
\begin{itemize}
\item 
\end{itemize}
\end{small}

\end{document}